% Change numbering to be at subsections but not higher up (not sections, chapters)
% This needs to be done before hyperref
\makeatletter
\def\@sect#1#2#3#4#5#6[#7]#8{%
  \refstepcounter{#1}%
  \ifnum #2<\c@secnumdepth
    \let\@svsec\@empty
  \else
    \protected@edef\@svsec{\@seccntformat{#1}\relax}%
  \fi
  \@tempskipa #5\relax
  \ifdim \@tempskipa>\z@
    \begingroup
      #6{%
        \@hangfrom{\hskip #3\relax\@svsec}%
          \interlinepenalty \@M #8\@@par}%
    \endgroup
    \csname #1mark\endcsname{#7}%
    \addcontentsline{toc}{#1}{%
      \ifnum #2<\c@secnumdepth \else
        \protect\numberline{\csname the#1\endcsname}%
      \fi
      #7}%
  \else
    \def\@svsechd{%
      #6{\hskip #3\relax
      \@svsec #8}%
      \csname #1mark\endcsname{#7}%
      \addcontentsline{toc}{#1}{%
        \ifnum #2<\c@secnumdepth \else
          \protect\numberline{\csname the#1\endcsname}%
        \fi
        #7}}%
  \fi
  \@xsect{#5}}

\def\@makechapterhead#1{%
  \vspace*{50\p@}%
  {\parindent \z@ \raggedright \normalfont
    \interlinepenalty\@M
    \Huge \sffamily\bfseries #1\par\nobreak
    \vskip 40\p@
}}

\@addtoreset{subsection}{section}
\makeatother

\usepackage{enumitem, amsmath, amssymb, tikz, fancyhdr, stackengine, graphicx, pdfpages, hyperref, microtype}
\renewcommand{\sectionmark}[1]{\markright{#1}{}}

\usepackage[top=18mm, inner=9mm, outer=10mm, bottom=17mm, headheight=14.5pt, headsep=3mm, footskip=7mm]{geometry}

    \usepackage[english, russian]{babel}    %% загружает пакет многоязыковой вёрстки
    \usepackage{fontspec}                   %% подготавливает загрузку шрифтов 
    \defaultfontfeatures[\rmfamily, \sffamily]{Ligatures={TeX}}  %% свойства шрифтов по умолчанию
    % \setmainfont[Scale=.9, BoldFont={[PTF75F.ttf]}, BoldItalicFont={[PTF76F.ttf]}, ItalicFont={[PTF56F.ttf]}, Ligatures={TeX,Historic}]{[PTF55F.ttf]} %% задаёт основной шрифт документа
    \setmainfont{PT Serif}
    % \setsansfont[Scale=.9, BoldFont={[PTS75F.ttf]}, BoldItalicFont={[PTS76F.ttf]}, ItalicFont={[PTS56F.ttf]}, Ligatures={TeX,Historic}]{[PTS55F.ttf]} %% задаёт шрифт без засечек
    \setsansfont{PT Sans}
    % \setmonofont{consola.ttf}              %% задаёт моноширинный шрифт 
    \setmonofont{Consolas}

    \usetikzlibrary{shapes.geometric}

    \usepackage{unicode-math}
    \setmathfont{[latinmodern-math.otf]}
    \setmathfont[range=\mathit/{latin,Latin}]{PT Serif Italic}
    %\setmathfont[range=\mathit/{greek,Greek}]{Arno Pro}    
    \setmathfont[range=up]{PT Serif}
    \setmathfont[range="2264-"2265]{PT Serif}
    \setmathfont[range="003C-"003E]{PT Serif}
	
    \usepackage{verbatim}
    \usepackage{import}

    \setlist[enumerate,1]{nolistsep}
    \setlist[itemize,1]{nolistsep}

    % русские названия заголовков разделов 
    \newcommand{\problemsname}{Задачи}
    \newcommand{\solutionsname}{Решения}
    \newcommand{\inputname}{Формат входных данных}
    \newcommand{\outputname}{Формат выходных данных}
    \newcommand{\examplename}{Пример входного и выходного файлов}
    \newcommand{\examplesname}{Примеры входного и выходного файлов}
    \newcommand{\commentsname}{Комментарии}
    \newcommand{\interactionname}{Протокол взаимодействия}
    \newcommand{\evaluationname}{Описание системы оценивания}
    \newcommand{\InputFileName}{input.txt}
    \newcommand{\OutputFileName}{output.txt}
    \makeatletter
    \def\kw@Explanation{Пояснение к примеру}
    \def\kw@Explanations{Пояснения к примерам}
    \makeatother
    
    % заголовки разделов и~столбцов таблицы примеров
    \newcommand{\problempartheading}[1]{\par\medskip\noindent\textbf{\sffamily #1}\par\penalty10000\noindent}    
    \newcommand{\InputFile}{\problempartheading{\inputname}}
    \newcommand{\OutputFile}{\problempartheading{\outputname}}
    \newcommand{\Comments}{\problempartheading{\commentsname}}
    \newcommand{\Scoring}{\problempartheading{\evaluationname}}
    \newcommand{\Interaction}{\problempartheading{\interactionname}}
    \newcommand{\Example}{\par\medskip\noindent\textbf{\sffamily\examplename}\raisebox{-2pt}{\strut}\par\penalty10000\noindent}
    \newcommand{\Examples}{\par\medskip\noindent\textbf{\sffamily\examplesname}\raisebox{-2pt}{\strut}\par\penalty10000\noindent}
    \newcommand{\Note}{\Comments}
    \newcommand{\Explanation}{\problempartheading{\kw@Explanation}}
    \newcommand{\Explanations}{\problempartheading{\kw@Explanations}}

	  % русские знаки нестрогих неравенств
    \renewcommand{\geq}{\geqslant}
    \renewcommand{\leq}{\leqslant}
    \renewcommand{\ge}{\geq}
    \renewcommand{\le}{\leq}

    % штрафы за разрыв строки по оператору
    \binoppenalty=10000
    \relpenalty=10000

    % задача = \begin{problem}{name}{in}{out}{TL}{ML}
    \newenvironment{problem}[6][\unskip]{%
\subsection[#2 \emph{#1}]{#2 ~~{\mdseries\itshape #1}}%
\renewcommand{\InputFileName}{#3}%
\renewcommand{\OutputFileName}{#4}%
}{}
    \newenvironment{tutorial}[1]{\subsection{#1}}{}
    \renewcommand{\thesubsection}{\Alph{subsection}.}

    \makeatletter

    % ширина таблицы примеров - по умолчанию равна ширине текста минус 2 красных строки
    \newlength{\ex@mpwidth}
    \setlength{\ex@mpwidth}{\textwidth}
    \advance\ex@mpwidth-2\parindent
    \advance\ex@mpwidth-4\tabcolsep

    % ширины колонок ввода, вывода и~примеров для таблиц примеров
    \newlength{\ex@mpinpwidth}
    \newlength{\ex@mpoutwidth}
    \newlength{\ex@mpcomwidth}

    \makeatother

% -- Setup sizes --
\newlength{\thelinewidth}
\thelinewidth=\textwidth
\newlength{\exmpwidinf}
\newlength{\exmpwidouf}
\newlength{\exmpwidewid}
\newlength{\exmpthreewidinf}
\newlength{\exmpthreewidouf}
\newlength{\exmpthreewidnote}

\newif\ifintentionallyblankpages

\exmpwidinf=0.43\thelinewidth
\exmpwidouf=0.43\thelinewidth
\exmpwidewid=0.9\thelinewidth
\exmpthreewidinf=0.28\thelinewidth
\exmpthreewidouf=0.28\thelinewidth
\exmpthreewidnote=0.30\thelinewidth


    % во всех примерах нужно следить за разрывами строк, вставлять знак комментария после завершения команды \exmp во избежание ненужных разрывов
    % например \exmp{2 4}{3 5}%

    
    % таблица примеров с 2 колонками (ввод и~вывод)
% :FIXME:

% This is magic, which delete space after verbatiminput
\makeatletter
\addto@hook{\every@verbatim}{\topsep=0pt\relax}

\def\s@tm@cr@s{
    \def\widthin##1{\exmpwidinf=##1\relax}
    \def\widthout##1{\exmpwidouf=##1\relax}
    \def\stretchin##1{\advance\exmpwidinf by ##1\relax}
    \def\stretchout##1{\advance\exmpwidouf by ##1\relax}
    \@ifstar{
        \error Star must not be used in example environment any more
    }
}


\newenvironment{example}[1][]{
    \s@tm@cr@s#1
    \ttfamily\obeylines\obeyspaces\frenchspacing
    \newcommand{\exmp}[2]{
        \begin{minipage}[t]{\exmpwidinf}\rightskip=0pt plus 1fill\relax##1\medskip\end{minipage}&
        \begin{minipage}[t]{\exmpwidouf}\rightskip=0pt plus 1fill\relax##2\medskip\end{minipage}\\
        \hline
    }

    \newcommand{\exmpfile}[2]{
       \exmp{
          \verbatiminput{##1}
       }{
          \verbatiminput{##2}
       }%
    }


    \begin{tabular}{|l|l|}
        \hline
        \multicolumn{1}{|c|}{\bf\texttt{\InputFileName}}&
        \multicolumn{1}{|c|}{\bf\texttt{\OutputFileName}}\\
        \hline
}{
    \end{tabular}
}

    
% таблица примеров с 3 колонками (ввод, вывод, комментарий), три аргумента - доли колонок 
\newenvironment{examplecommented}[3]{
\setlength{\ex@mpinpwidth}{#1\ex@mpwidth}
\setlength{\ex@mpoutwidth}{#2\ex@mpwidth}
\setlength{\ex@mpcomwidth}{#3\ex@mpwidth}
\newcommand{\exmp}[3]{\begin{minipage}[t]{\ex@mpinpwidth}\raggedright\ttfamily ##1\medskip\end{minipage} &
                      \begin{minipage}[t]{\ex@mpoutwidth}\raggedright\ttfamily ##2\medskip\end{minipage} &
                      \begin{minipage}[t]{\ex@mpcomwidth} ##3\medskip\end{minipage}\\\hline%
}
\obeylines\obeyspaces\frenchspacing
\par\noindent\begin{tabular}{|l|l|l|}
\hline\multicolumn{1}{|c|}{\bf\texttt{\InputFileName}}&\multicolumn{1}{c|}{\bf\texttt{\OutputFileName}}&
\multicolumn{1}{c|}{\bf\texttt{\commentsname}}\\
\hline
}{\end{tabular}\smallskip}

    
    % таблица примеров в~1 колонку. Один пример верстается в~4 строки (считая заголовки) на всю ширину таблицы
\newenvironment{examplewide}{
\advance\ex@mpwidth2\tabcolsep 
\ttfamily\obeylines\obeyspaces\frenchspacing
\newcommand{\exmp}[2]{
    \begin{tabular}{|c|}
    \hline
    \multicolumn{1}{|c|}{\bf\texttt{\InputFileName}}\\
    \hline
    \begin{minipage}[t]{\ex@mpwidth}\rightskip=0pt plus 1fill\relax\raggedright ##1 \medskip\end{minipage}\\
    \hline
    \multicolumn{1}{|c|}{\bf\texttt{\OutputFileName}}\\
    \hline
    \begin{minipage}[t]{\ex@mpwidth}\rightskip=0pt plus 1fill\relax\raggedright ##2 \medskip\end{minipage}\\%
    \hline
    \end{tabular}\smallskip
}
}{}

\usepackage{etoolbox}
\patchcmd{\tableofcontents}{\@starttoc{toc}}{\thispagestyle{empty}\pagestyle{empty}\@starttoc{toc}}{}{}

\makeatother