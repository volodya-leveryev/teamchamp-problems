\begin{tutorial}{Робот-садовод}

Бамбук нужно срубить пополам столько раз, сколько раз можно ее длину поделить на два без остатка (заметим, что длина бамбука не может быть нулевой, иначе делить на два можно было бы бесконечно). 

Двоичное число делится на два, только если оно оканчивается на ноль, и при делении на два этот ноль с конца числа просто удаляется. Тогда количество делений на два без остатка~--- это количество нулей в~конце двоичной записи числа. Ответом будет общее количество нулей в концах всех заданных двоичных чисел.

\end{tutorial}
