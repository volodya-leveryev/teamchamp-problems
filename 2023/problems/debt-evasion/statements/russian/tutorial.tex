\begin{tutorial}{Уклонение от уплаты долгов}

Перед решением задачи отсортируем массив $W$, содержащий количество занятых денег по цветам курток, в~которых они были заняты.

Будем просматривать этот массив слева направо. Чтобы учесть похожесть цветов, просмотр будем выполнять при помощи двух границ (образующих <<окно>>), внутри которых разница в~цвете курток меньше $K$. Вычисленную сумму долга для окна будем корректировать при каждом сдвиге окна (добавлять суммы справа и вычитать суммы слева). Данный шаг будет иметь линейную трудоемкость.

При просмотре массива следует найти минимальную сумму в~окне. Вычислительная сложность данного способа решения будет иметь порядок $O(N \log_2 N)$ (учитывая операции по сортировки массива).

Стоит отметить, что ограничения на трудоемкость задачи позволяют находить границы <<окна>> каждый раз заново, используя бинарный поиск. Имея границы, можно достаточно быстро вычислить сумму денег в~<<окне>> (например, с~помощью массива префиксных сумм).

% Если для каждой куртки посчитать сколько денег в~ней было занято денег, и затем отсортировать куртки по возрастанию цвета, то получится что нам нужно найти минимум по суммам по скользящему окну. Пройдемся по курткам которые можно надеть по возрастанию и с~помощью указателей на левую и правую границу похожих курток, понятно что границы могут только возрастать, и поэтому они изменятся не более $n$ раз каждая. Также ограничения позволяют находить границы каждый раз заново с~помощью бинарного поиска. Имея границы можно легко найти сумму денег, например с~помощью массива префиксных сумм.

\end{tutorial}
