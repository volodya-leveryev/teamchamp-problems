\begin{problem}{Уклонение от уплаты долгов}{стандартный ввод}{стандартный вывод}{1 секунда}{256 мегабайт}

Известная пословица гласит: <<Не имей сто рублей, а имей сто друзей>>. Леонид Скорняков понял эту пословицу по своему, поэтому он постоянно берёт деньги в долг у своих друзей. Так как Леонид Скорняков сдал IQ-тест на 200 баллов, он берёт в долг только у друзей с плохим зрением. И если он брал в долг у друга в куртке цвета $A$, то этот друг не узнает его, если цвет куртки $B$, в которой он увидит Леонида Скорнякова, не похож на цвет $A$. Соответственно, если он не узнает его, то не сможет потребовать вернуть долг. 

Цвета $A$ и $B$ называются похожими, если $|A - B| \leq k$. 

Обычно Леонид Скорняков старается не выходить из дома, так как его могут узнать его друзья, поджидающие его у выхода из дома, и потребовать вернуть деньги. А он не может им отказать, так как иначе они перестанут с ним дружить. Но в этот раз ему очень сильно понадобилось выйти наружу, помогите ему подобрать цвет куртки такой, чтобы узнавшие его друзья потребовали наименьшее количество денег. 

Во всех своих куртках он уже брал долг, то есть он может выбрать только такой цвет куртки, в~которой он брал долг.

\InputFile
Первая строка содержит два целых числа $n$ и $k$ ($1 \leq n \leq 10^5, 0 \leq k \leq 10^9$) ~--- количество человек, которым Леонид Скорняков должен и константа, определяющая похожесть цветов, соответственно. 

Вторая строка содержит $n$ целых чисел $w_1, w_2, \dots, w_n$ ($1 \leq w_i \leq 10^9$).  $w_i$~--- это количество денег, которое Леонид Скорняков задолжал $i$-му заемщику. 

Третья строка содержит $n$ целых чисел $c_1, c_2, \dots, c_n$ ($1 \leq c_i \leq 10^9$). $c_i$~--- это цвет куртки, в~которой был Леонид Скорняков, когда занял деньги у $i$-го заемщика.

\OutputFile
Выведите одно число $W$ --- наименьшее количество денег, которое вернёт Леонид Скорняков при надевании правильной куртки.

\Example

\begin{example}
\exmpfile{example.01}{example.01.a}%
\end{example}

\end{problem}

