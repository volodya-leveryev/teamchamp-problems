\begin{tutorial}{Простая игра}

В рамках данного разбора будем называть числа, на которые Демон ответил <<NO>>~--- <<плохими>>, а числа, на которые ответил <<YES>>~--- <<хорошими>>.

Начнем просматривать целые числа с 2 до 9999. Текущее число обозначим $x$.

Чтобы повысить эффективность алгоритма, следует проверять у Демона все простые числа. Поэтому каждое очередное число $x$ сначала нужно проверить на простоту. Если оно не делится на все найденные до этого простые числа~--- значит оно простое, и его нужно добавить в~список простых чисел.

С другой стороны, если $x$~--- составное, то нужно проверить следующие условия:

\begin{itemize}
\setlength{\itemsep}{0pt}%
%\setlength{\parskip}{0pt}%
\item Если $x$ делится хотя бы на одно <<плохое>> число~--- пропускаем его и переходим к~следующему.
\item Если $x$ не делится хотя бы на одно <<хорошее>> число~--- пропускаем его и переходим к следующему.
\end{itemize}

Теперь отдадим $x$ на проверку Демону. Если ответ <<YES>>~--- добавим его к списку <<хороших>> чисел, если ответ <<NO>>~--- добавим его к списку <<плохих>> чисел.

Например, если Демон загадал число 6, то действия должны быть следующие:

\begin{itemize}
\setlength{\itemsep}{0pt}%
%\setlength{\parskip}{0pt}%
\item Проверим простое число 2, Демон ответит <<YES>>, значит, все последующие составные числа должны быть кратны 2.
\item Проверим простое число 3, Демон ответит <<YES>>, значит, все последующие составные числа должны быть кратны 3.
\item Составное число 4 пропустим, т.\,к. оно не делится на 3.
\item Проверим простое число 5, Демон ответит <<NO>>, значит, все последующие составные числа не должны делиться на 5.
\item Проверим 6, Демон ответит <<OK>>, программа должна завершить работу.
\end{itemize}

\end{tutorial}
