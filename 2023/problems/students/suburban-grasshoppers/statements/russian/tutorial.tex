\begin{tutorial}{Электричка кузнечиков}

Здесь можно использовать жадный алгоритм следующим образом.

Отсортировать длины прыжков кузнечиков в порядке убывания.

Пройти по отсортированным длинам прыжков и поочередно попытаться заселить кузнечика на свободное место, начиная с первого.

Если для текущего кузнечика нашлось свободное место, продолжить рассадку со следующим кузнечиком. Если для текущего кузнечика свободного места не нашлось, то он не сможет занять место в электричке и алгоритм завершается.

Если все кузнечики успешно заселились в электричку, то алгоритм завершается успешно.

Очевидно, что если алгоритм найдет какую-нибудь рассадку, то она корректна.

Теперь докажем, что если существует корректная рассадка, то алгоритм обязательно найдёт какую-нибудь корректную рассадку. Рассмотрим самое дальнее занятое место. Утверждается, что наш алгоритм займет самое дальнее место не дальше, чем самое дальнее место какой-нибудь корректной рассадки. Пусть это не так, и наш алгоритм ушел чуть дальше положенного. Рассмотрим путь этого кузнечика --- все предшествующие ему кузнечики на его пути имеют прыжок больше или равный ему, причем все прыжки, которые больше, делятся на его прыжок (степени двойки). Тогда отсюда следует факт, что на этом пути содержатся \textbf{только и только те} кузнечики, которые имеют прыжок больше или равный прыжку нашего кузнечика. Получили противоречие с тем, что существует рассадка, которая имеет меньшее самое дальнее занятое место.

\end{tutorial}
