\begin{problem}[(Иванов В.\,В.)]{Прямой путь}{стандартный ввод}{стандартный вывод}{1 секунда}{256 мегабайт}

Золотоискатель Петр объездил весь мир и, наконец, собрал все древние фрагменты с загадочными надписями. Он был уверен, что это подсказки и если их разгадать, то они приведут к секретному хранилищу с золотом! Он обратился к эксперту-расшифровщику, и вместе они смогли разгадать тайну!

На каждом из фрагментов было по четыре числа: первое всегда было целым положительным числом, второе и четвертое были числами с~двумя знаками после запятой, а третье число всегда было $1$ или $2$.

Оказалось, что первое число означает порядковый номер фрагмента. Второе число лежит в диапазоне от $0$ до $180$ и означает градус угла поворота. Третье число означает направление поворота, где $1$~--- поворот налево, а $2$~--- поворот направо. Четвертое число лежит в диапазоне от $0$ до $10^5$ и означает расстояние, которое надо пройти в этом направлении. 

Вместе они сразу пришли к выводу, что используется декартова система координат и начинать надо с места, где был найден фрагмент с порядковым номером $1$. Это место имеет координаты $(0, 0)$, а первый поворот нужно делать относительно оси абсцисс.

\InputFile
В первой строке записано количество фрагментов $N$, где $1 \leq N \leq 10000$. Затем в $N$ строках описаны фрагменты.

\OutputFile
Выведите через пробел координаты места $X$ и $Y$ с точностью до двух знаков после запятой, где находится секретное хранилище с золотом.

\Example

\begin{example}
\exmpfile{example.01}{example.01.a}%
\end{example}

\Note
В качестве разделителя целой и дробной части во всех вещественных числах используется символ точки.

\end{problem}

