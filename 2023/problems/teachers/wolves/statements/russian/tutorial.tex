\begin{tutorial}{Волки и медведи}

Пусть $n$ и $m$~--- соответственно, количество волков и медведей. Очевидно, что если изначально волков не менее, чем увеличенное в четыре раза число медведей, то все они останутся живы, это и будет ответом. С другой стороны, если волков не больше, чем удвоенное количество медведей, то все они погибнут, и ответом будет $-1$.

Иначе помещаем эти числа в~очередь схваток. Пока очередь не пуста, выполняем следующие действия:
\begin{itemize}
\item Извлекаем первую пару из очереди, текущее количество волков и~медведей. 
\item Моделируем все варианты исхода одной битвы. Вычисляем максимальное количество четверок волков $kch=n$ div 4 (здесь операция \textbf{div} означает деление нацело). Далее рассматриваем ситуации для $kch, kch-1, kch-2, \dots, 0$ и каждый раз вычисляем количество троек волков среди оставшихся $ktr = (n - kch \cdot 4)$ div 3. Это тоже самое, что определить максимальное количество убитых медведей. Для каждой ситуации пересчитываем новое количество волков и медведей после схватки по формулам: $m1=m-kch-ktr$; $n1=n-ktr-2\cdot m1$. Эти числа помещаем в очередь.\\
Параллельно поддерживаем текущий максимум $mv$ среди оставшихся волков, если число медведей в этой конкретной схватке окажется нулем.  
\end{itemize}
После окончания обработки очереди ответом будет $mv$.



\end{tutorial}
