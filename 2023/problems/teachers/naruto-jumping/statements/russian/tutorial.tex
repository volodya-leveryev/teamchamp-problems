\begin{tutorial}{Зиплайн по дороге из Дьаргалаха}

{
  \sloppy
  В числовой последовательности $\{h_i\}$ требуется найти отрезок максимальной длины, который является строго убывающей последовательностью, возможно, после увеличения одного элемента $h_m$ на некоторое $d\geq 0$:
  $$h_i > h_{i+1} > \ldots > h_m + d > h_{m+1} > \ldots > h_{j-1},$$
  где $i\leq m < j$.
}

Если наращивание дерева не требуется, то $d=0$ и перед нами максимальный (не расширяемый ни в одну из сторон) строго убывающий отрезок. 

Если же наращивание $d>0$ необходимо, то без него отрезок строго убывающим не будет, то есть $m<j-1$ и $h_m \leq h_{m+1}$. Таким образом, данному отрезку в исходной последовательности соответствуют два строго убывающих, причем наращиваемым элементом завершается первый из них:
$$h_i > h_{i+1} > \ldots > h_m \quad \mbox{ и } \quad  h_{m+1} > \ldots > h_{j-1},$$
и $h_{m-1} > h_{m+1}$, если $m > i$. Так как зиплайн имеет максимальную длину, то второй отрезок~--- максимальный строго убывающий, а для первого фрагмента предыдущее условие означает, что он или представляет собой максимальный строго убывающий отрезок длины $\geq 2$, или же состоит из одного числа $h_i=h_m \leq h_{m+1}$.

Два максимальных отрезка строгого убывания в последовательности, очевидно, не пересекаются. Таким образом, достаточно перебрать все отрезки строгого убывания (число действий пропорционально $N$), запоминая предыдущий. К каждому отрезку, начиная со второго, можно добавить или последнее дерево из предыдущего отрезка, или весь предыдущий отрезок длины $\geq 2$, если его предпоследний элемент больше, чем первый элемент текущего.

\end{tutorial}
