\begin{problem}[(Павлов А.\,В.)]{Зиплайн по дороге из Дьаргалаха}{стандартный ввод}{стандартный вывод}{1 секунда}{256 мегабайт}

Мичил открывает новый аттракцион на дороге из Дьаргалаха в Кулусуннах. Это зиплайн, состоящий из тросов, по которым посетители скатываются на специальных роликовых подвесах под действием силы тяжести. Проехав от Дьаргалаха в Кулусуннах, Мичил измерил высоты всех растущих вдоль дороги деревьев. Тросы должны последовательно (скатываться будут в том же направлении, в котором проводились измерения) соединять деревья на некотором отрезке дороги, где каждое следующее дерево должна быть строго ниже предыдущего. 

Разумеется, Мичил хочет построить зиплайн максимальной длины. Для этого он даже может нарастить одно дерево до любой нужной высоты (на большее не хватит материалов). Помогите ему понять, какой наибольшей длины зиплайн он сможет построить.

\InputFile
В первой строке задано количество $N$ деревьев ($1 \leq N \leq 10^6)$. Во второй строке через пробел записаны $N$~положительных вещественных чисел $h_i$~--- их высоты ($0 < h_i \leq 10^6$).

\OutputFile
Одно целое число~--- максимальная длина зиплайна, который сможет построить Мичил, возможно, нарастив одно из деревьев. Длиной называется число тросов в зиплайне, например, если последовательно соединены три дерева, то зиплайн имеет длину два.

\Example

\begin{example}
\exmpfile{example.01}{example.01.a}%
\end{example}

\Note
В примере одно из деревьев высоты 1 можно нарастить, как показано на рисунке:\\
\centerline{\includegraphics{zipline.png}}

\end{problem}

