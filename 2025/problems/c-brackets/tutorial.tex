\begin{tutorial}{Скобки}

В данной задаче необходимо заметить следующее:
\begin{itemize}
  \item Для того чтобы можно было составить правильную скобочную последовательность, количество открывающихся скобок должно быть равно количеству закрывающихся;
  \item Если существует ответ на эту задачу, то он всегда будет начинаться и заканчиваться открывающейся скобкой, и заканчиваться закрывающейся.
\end{itemize}

Теперь рассмотрим варианты скобочных последовательностей, которые могут передаваться во входных данных:
\begin{enumerate}
  \item Только открывающиеся скобки \[
      \underbrace{\texttt{((}\cdots\texttt{(((}}_{n \text{ скобок}}
    \]
    В таком случае легко заметить, что их можно расположить в начале результирующей последовательности, при этом порядок расположения не важен;
  \item Только закрывающиеся скобки  \[
      \underbrace{\texttt{))}\cdots\texttt{)))}}_{m \text{ скобок}}
    \]
    Аналогично первому варианту, их можно разместить в конце итоговой последовательности в любом порядке;
  \item Сначала закрывающиеся, потом открывающиеся скобки \[
      \underbrace{\texttt{))}\cdots\texttt{)))}}_{m \text{ скобок}}\underbrace{\texttt{((}\cdots\texttt{(((}}_{n \text{ скобок}}
    \]
    Перед расположением данного варианта последовательности необходимо, чтобы перед ней было как минимум $m$ открывающихся скобок без пары;
  \item Произвольный набор скобок \\
    Легко заметить, что если удалить все правильные скобочные подпоследовательности, то результатом будет последовательность типа $1$, $2$ или $3$. В решении задачи можно упростить последовательность, удалив все подпоследовательности \texttt{()} пока это возможно.
\end{enumerate}

После упрощения всех последовательностей во входных данных, можно вывести следующий алгоритм, который позволит найти ответ:
\begin{enumerate}
  \item Разместить в начале строки все последовательности типа $1$;
  \item Разделить последовательности типа $3$ на две группы:
  \begin{enumerate}
    \item количество открывающихся скобок больше или равно количеству закрывающихся,
    \item количество закрывающихся скобок больше, чем открывающихся;
  \end{enumerate}
  \item Каждую группу отсортировать по возрастанию количества закрывающихся скобок;
  \item Разместить первую группу, затем вторую группу в получившемся порядке;
  \item Разместить в конце строки все последовательности типа $2$;
  \item Убедиться, что получившаяся последовательность является правильной.
\end{enumerate}

Если получившаяся последовательность не является правильной, значит ответа нет. Иначе эта последовательность является ответом на задачу. Необходимо учитывать, что в ответе нужно вернуть номера строк входных данных, поэтому при сортировке и манипуляциях с последовательностью необходимо сохранять эти данные.

\end{tutorial}
