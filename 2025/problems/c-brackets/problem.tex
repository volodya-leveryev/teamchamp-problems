\begin{problem}{Скобки}{стандартный ввод}{стандартный вывод}{1 секунда}{256 мегабайт}

Айтал и Григорий~--- школьники и настоящие энтузиасты программирования. После уроков они частенько зависают в кабинете информатики, решая олимпиадные задачи и обсуждая тонкости алгоритмов.

Однажды, готовясь к региональной олимпиаде, они придумали игру: каждый из них написал несколько строк $S_i$, состоящих только из круглых скобок `(' и `)'. В итоге у них получилось $N$ строк~--- не обязательно правильных, но каждая из них может быть частью большой правильной скобочной последовательности.

\textit{Правильная скобочная последовательность}~--- это такая строка, в которой:
\begin{itemize}
\item каждая открывающая скобка `(' имеет соответствующую ей закрывающую скобку `)';
\item открывающие скобки `(' должны быть записаны раньше своих закрывающих скобок `)';
\item нет одиночных скобок, то есть таких, у которых нет пары.
\end{itemize}

Айтал предложил интересную задачу: можно ли упорядочить эти $N$ строк в каком-либо порядке так, чтобы при их склеивании (то есть соединении в одну большую строку) получилась правильная скобочная последовательность?

Григорий задумался... А ты сможешь помочь им?

\InputFile
В первой строке записано единственное целое положительное число $N$ ($1 \leq N \leq 4\cdot10^4$)~--- количество строк со скобочными последовательностями.

Далее идут $N$ строк, в каждой из которых содержится скобочная последовательность $S_i$.

Гарантируется, что $\displaystyle \sum_{i=1}^{N} \left| S_i \right| \leq 10^5$, где $\left| S_i \right|$~--- длина $i$-й скобочной последовательности.

\OutputFile
В первой строке нужно вывести <<Yes>>, если существует хотя бы один вариант составления правильной скобочной последовательности, использующей все $N$ строк, либо <<No>> в противном случае.

Если ответ существует, то во второй строке вывести последовательность номеров строк, в соответствии с которой нужно расставить скобочные последовательности. Строки нумеруются с 1 в порядке появления во входных данных. Если ответов несколько, выведите любой.

\Example

\begin{example}
\exmpfile{example.01}{example.01.a}%
\end{example}

\end{problem}

