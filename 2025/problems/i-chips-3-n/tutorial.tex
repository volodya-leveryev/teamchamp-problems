\begin{tutorial}{Фишки}

Есть множество способов построения искомой расстановки. Мы покажем, почему ответ всегда существует. Доказательство проведём с помощью математической индукции:
\begin{itemize}
\item \textbf{База:} при $n=1$ очевидно, что любая такая доска $3\times1$ с одной фишкой каждого цвета удовлетворяет искомой расстановке.
\item \textbf{Предположение:} предположим, что любую расстановку фишек на доске $3\times n$ можно привести к искомой.
\item \textbf{Переход:} в доске $3\times (n + 1)$ сформируем первый столбец, состоящий из всех трёх цветов, а оставшиеся $n$ столбцов переставим, следуя предположению (ведь у нас останется по $n$ штук фишек каждого цвета).
\end{itemize}

\end{tutorial}
