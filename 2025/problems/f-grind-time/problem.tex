\begin{problem}{Время гринда}{стандартный ввод}{стандартный вывод}{1 секунда}{256 мегабайт}

Коля продолжает играть в игры от компании TimurKul Games. Недавно он скачал игру <<Время приключений 2: в погоне за деньгами>>, которая, к сожалению, отличается своей реиграбельностью.

Игру можно представить как $n$ последовательных локаций, каждая из которых характеризуется значением $a_i$~--- энергией, требуемой для прохождения локации. Изначально у Коли имеется $k$ единиц энергии. Игра проходит следующим образом:

\begin{enumerate}
\item Вначале игрок находится в локации $i = 1$.
\item Игрок может пройти $i$-ю локацию, если у него имеется хотя бы $a_i$ энергии. В противном случае игра заканчивается.
\item После прохождения $i$-й локации от энергии игрока отнимается $a_i$, а игрок перемещается в~локацию $i + 1$. Если после прохождения последней локации, у игрока все еще имеется энергия, игрок перемещается к первой локации $i = 1$, и на свое усмотрение усложняет одну из локаций, то есть к выбранному игроком значению $a_i$ прибавляется величина $c$.
\end{enumerate}

Проблема в том, что значение $c$ выбирается самим игроком. Коля хочет, чтобы выполнялось неравенство $0 \leq c \leq 10^9$ и чтобы максимально возможное количество локаций, которое возможно пройти, было не меньше $m$. Если подходящих значений $c$ несколько, то он предпочитает самое максимальное. Помогите Коле узнать, какое значение $c$ ему выбрать.

\InputFile
Первая строка содержит три целых числа $n$, $m$ и $k$ ($1 \leq n \leq 10^5; 0 \leq m \leq 10^9; 0 \leq k \leq 10^9$)~--- количество локаций, требуемое количество пройденных локаций и начальное количество энергии соответственно. Гарантируется, что у заданного $m$ существует хотя бы одно $c$, которое удовлетворяет вышеописанным условиям.

Вторая строка содержит $n$ целых чисел $a_1,a_2,...,a_n$($1 \leq a_i \leq 10^4$)~--- значения энергии, необходимые для прохождения локаций.

\OutputFile
Выведите максимальное из подходящих значений $c$.

\Examples

\begin{example}
\exmpfile{example.01}{example.01.a}%
\exmpfile{example.02}{example.02.a}%
\exmpfile{example.03}{example.03.a}%
\end{example}

\Note
В первом примере сначала $a = [2, 3]$, и Коля находится в локации $1$. Он может пройти обе локации, потратив на это $5$ единиц энергии. После прохождения локации $2$ Коля должен прибавить к какой-либо локации $c = 2$. Пусть он выберет $a_2$. После всего этого Коля находится в локации $1$, имеет энергию $k = 9$ и $a = [2, 5]$. Коля может снова пройти обе эти локации, потратив на это $7$ единиц энергии и прибавив к $a_2$ $2$. Наконец, Коля снова находится в локации $1$, имеет энергию $k = 2$ и $a = [2, 7]$. Коля может пройти первую локацию, но застрянет на второй. В итоге он прошёл $5$ локаций, что не меньше  $m = 5$. Можно показать, что $c$ не может быть больше $2$.

Во втором примере Коля, ничего не делая, уже выполняет свой минимум по пройденным локациям, из-за чего значение $c$ можно выбрать любое. Поскольку Коля предпочитает максимальное из возможных $c$, ответом является $10^9$.

\end{problem}

