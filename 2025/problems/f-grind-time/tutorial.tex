\begin{tutorial}{Время гринда}

Есть 2 способа решить данную задачу: с помощью бинарного поиска, с помощью формулы.

Первый способ заключается в том, что если при каком-то $c$ количество локаций, которое может пройти Коля, не меньше $m$, то при $c - 1$ данное условие тоже выполняется. Также замечаем, что начиная с какого-то $c$ количество локаций, которое пройдёт Коля, будет меньше $m$. Всё это значит, что $c$ можно перебрать с помощью бинарного поиска.

Далее, мы должны быстро вычислить количество локаций, которое Коля сможет пройти при каком-то $c$. Сначала сделаем вывод, что Коле выгоднее всего прибавить $c$ к последней локации. Если мы будем обрабатывать локацию за локацией, пока энергия не кончится, то наше решение не пройдёт по времени (например, если массив $a$ представляет из себя $100000$ единичек). Есть очень лёгкий способ оптимизировать процесс вычисления пройденных локаций: если мы находимся в локации $1$ и мы можем пройти все локации, то давайте не будем поштучно обрабатывать все локации, а просто добавим к счётчику $n$ и вычтем из энергии сумму $a_i$ всех локаций и <<индекс>> текущего прохождения. Это будет работать быстро по той причине, что каждое новое прохождение характеризуется увеличением суммы всех $a_i$ хотя бы на 1. Это значит, что примерно через 45000 прохождений всех локаций даже если $a = [1]$, то энергия Коли всё равно закончится.

Учтите, что $c = 0$ не надо проверять, поскольку условие гарантирует, что при $c = 0$ Коля всегда сможет пройти $m$ локаций.

Для решения задачи с помощью второго способа, найдём количество энергии, требуемое для того, чтобы пройти $m$ локаций при $c = 0$. Данное значение равно (индексация с нуля):

\begin{equation}
\left( a_0 + a_1 + ... + a_{n-1} \right) \cdot \left\lfloor \frac{m}{n} \right\rfloor + a_0 + a_1 + ... + a_{m \bmod{n}} %\tag{1}
\end{equation}

Далее, если мы будем применять операцию только для последней локации, то количество дополнительной энергии можно выразить с помощью следующей формулы:

\begin{equation}
c \cdot \frac{\left( \left\lfloor \frac{m}{n} \right\rfloor - 1 \right) \cdot \left\lfloor \frac{m}{n} \right\rfloor}{2} %\tag{2}
\end{equation}

С помощью (1) и (2), выразим условие, выполнение которого означает, что $c$ подходит:

\begin{equation}
\left( a_0 + a_1 + ... + a_{n-1} \right) \cdot \left\lfloor \frac{m}{n} \right\rfloor + a_0 + a_1 + ... + a_{m \bmod{n}} + c \cdot \frac{\left( \left\lfloor \frac{m}{n} \right\rfloor - 1 \right) \cdot \left\lfloor \frac{m}{n} \right\rfloor}{2} \leq k %\tag{3}
\end{equation}

Пусть $k' = k - \left( 1 \right)$, а $\displaystyle b = \frac{\left( \left\lfloor \frac{m}{n} \right\rfloor - 1 \right) \cdot \left\lfloor \frac{m}{n} \right\rfloor}{2}$. Вычтем из обеих частей (1):

\begin{equation}
c \cdot b \leq k' %\tag{4}
\end{equation}

Пусть Коля может пройти все локации хотя бы 2 раза. Тогда (4) равносильно:

\begin{equation}
c \leq \frac{k'}{b} %\tag{5}
\end{equation}

Наконец, найдём ответ. Если Коля не сможет пройти все локации хотя бы 2 раза, то ответом является $1\,000\,000\,000$, поскольку от количество требуемой энергии не зависит от $c$. В противном случае ответом является:

\begin{equation}
c = \left\lfloor \frac{k'}{b} \right\rfloor %\tag{6}
\end{equation}

\end{tutorial}
