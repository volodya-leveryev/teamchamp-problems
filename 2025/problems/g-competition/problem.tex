\begin{problem}{Соревнование }{стандартный ввод}{стандартный вывод}{1 секунда}{256 мегабайт}

В зимних соревнованиях по рыбной ловле участники ловили рыбу в $n$ лунках, расположенных в один ряд. По окончании контрольного времени жюри подсчитало количество рыб, пойманных в~каждой лунке.

Оказалось, что в любых двух соседних лунках количество рыб отличается ровно на единицу, а в одной из лунок поймали $m$ рыб. Общее количество всех пойманных рыб может быть достаточно большим и принимать различные значения. Например, если лунок $4$ и в одной из них поймали $3$ рыбы, то суммарное количество пойманных рыб может принимать значения: $6, 8, 10, 12, 14, 16, 18$.

Помогите жюри определить, сколько различных значений может принимать общее количество пойманных рыб?

\InputFile
В единственной строке записаны два целых неотрицательных числа $n$ и $m$, разделенных пробелом ($0 \leq m \leq 10^6$, $1 \leq n \leq 10^6$).

\OutputFile
Программа должна напечатать одно число~--- количество различных значений общего количества рыб.

\Examples

\begin{example}
\exmpfile{example.01}{example.01.a}%
\exmpfile{example.02}{example.02.a}%
\end{example}

\end{problem}

