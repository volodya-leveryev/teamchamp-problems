\begin{tutorial}{Соревнование }

Обозначим
\begin{itemize}
\item SumMax(m,n) --- максимальная сумма пойманных рыб на n лунках, если в одной из лунок поймали m рыб;
\item SumMin(m,n) --- минимальная сумма пойманных рыб на n лунках, если в одной из лунок поймали m рыб.
\end{itemize}

Очевидно, что максимальная сумма будет равна

\texttt{SumMax(m,n) = m + (m+1) + (m+2) + \dots + (m+n-1) = (2m+n-1)*n/2.}

Вычисление минимальной суммы требует рассмотрения двух вариантов:
\begin{itemize}
\item  $m < n$: \texttt{SumMin(m, n) = m +(m-1) + (m-2) + \dots + 1 + 0 +1 + 0 + 1 + \dots = SumMax(0, m+1) + (n -- m) / 2;}
\item $m \ge n$: \texttt{SumMin(m,n) = m +(m-1) + (m-2) + \dots + m -- n + 1)= SumMax(m -- n + 1, n)}
\end{itemize}

Всевозможные суммы для четного числа n всегда будут четными или нечетными, т.к. суммы всех пар нечетны, а не пересекающихся пар либо четно, либо нечетно (см. пример). Для нечетного числа n суммы могут быть как четными, так и нечетными. Эти суммы можно разбить на два множества, с четными и нечетными элементами. Первая максимальная сумма вычислена. 

Вычислим вторую максимальную сумму с другой четностью элементов:

\texttt{MaxSum = (m+1) + m + (m+1) + \dots + (m+n-2) = m + 1 + SumMax(m,n-1)}

минимальная сумма для m > 0:

\texttt{MinSum = (m-1) + m + (m-1) + \dots + (m+n-3) = m - 1 + SumMin(m, n-1)}

для m = 0:

\texttt{MinSum = 1 + 0 + 1 + \dots + 0 + 1 = 1 + SumMin(0, n-1)}

Для четного n:

\texttt{ans = (SumMax(m,n) - SumMin(m,n))/2 + 1;}

для нечетног n:

\texttt{ans = ans + (MaxSum - MinSum)/2 + 1;}

\end{tutorial}
