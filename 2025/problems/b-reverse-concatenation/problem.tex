\begin{problem}{Обратная конкатенация}{стандартный ввод}{стандартный вывод}{1 секунда}{256 мегабайт}

Тимуру даётся массив $a$ длины $n$, состоящий из целых двузначных чисел. За одну операцию он может сделать следующее:

\begin{itemize}
\item Выбрать какие-то два числа с индексами $i$ и $j$ ($1 \leq i < j \leq n$);
\item Удалить $a_i$ из массива, а $a_j$ заменить на число вида $a_ja_i$.
\end{itemize}

Иными словами, к правому числу дописывается левое, затем левое число удаляется. Например, если $a_i=23$, а $a_j=31$, то после операции $a_i$ удаляется, а $a_j=3123$. Очевидно, что после $n-1$ операций останется лишь одно число.

Определите максимальное число, которое Тимур может получить в качестве оставшегося.

\InputFile
Первая строка содержит целое число $n$ ($1 \leq n \leq 10^5$)~--- количество элементов в массиве.

Вторая строка содержит $n$ целых чисел $a_1,a_2,...,a_n$ ($10 \leq a_i \leq 99$).

\OutputFile
Выведите максимальное число, которое можно получить в качестве последнего.

\Examples

\begin{example}
\exmpfile{example.01}{example.01.a}%
\exmpfile{example.02}{example.02.a}%
\exmpfile{example.03}{example.03.a}%
\exmpfile{example.04}{example.04.a}%
\end{example}

\end{problem}

