\begin{tutorial}{Обратная конкатенация}

Сначала сделаем наблюдение, что последнее число по-любому будет стоять первым в конечном числе. Далее, заметим, что порядок расположения всех остальных чисел после последнего числа может быть любым. Это значит, что наша задача - расположить числа $a_1, a_2, ..., a_{n-1}$ в таком порядке, чтобы конечное число было максимальным.

Сравним два возможных конечных числа $b$ и $c$. Представим их в виде $b_1b_2...b_{n - 1}$ и $c_1c_2...c_{n - 1}$. Пусть $b_i = c_i$ для $1 \leq i < x$ ($x \leq n - 1$) и $b_x < c_x$. В таком случае, очевидно, что $b < c$. Случай при $b_x > c_x$ рассматривается аналогично. Это означает, что мы можем жадно ставить в качестве очередного числа максимальный из неиспользованных чисел.

Пусть из $a_1, a_2, ..., a_{n-1}$ мы можем получить максимальное число $m_1m_2...m_{n-1}$. Ответом будет $a_nm_1m_2...m_{n-1}$.

\end{tutorial}
