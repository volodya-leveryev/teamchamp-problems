\begin{problem}{Радикал}{стандартный ввод}{стандартный вывод}{2 секунды}{256 мегабайт}

В самом сердце одной страны лежит магический камень, на котором выгравировано натуральное число $n$. С каждым новым указом синей партии число на камне увеличивается на \textit{радикал} этого числа. Формально, после каждого указа значение $n$ изменяется по правилу: $n \rightarrow n + \text{rad}(n)$.

\textit{Радикалом} числа называют произведение всех его различных простых делителей:
\begin{itemize}
\item $\text{rad}(1) = 1$;
\item Если $n = p_1^{\alpha_1}\cdot p_2^{\alpha_2}\cdot\ldots\cdot p_k^{\alpha_k}$, то $\text{rad}(n)=p_1 \cdot p_2\cdot\ldots\cdot p_k$.
\end{itemize}

Ваша задача~--- определить, какое число будет написано на камне после $k$ указов синей партии. Поскольку результат может быть очень большим, выведите его по модулю $10^9 + 7$.

\InputFile
В единственной строке содержатся два целых числа $n$ и $k$ ($1\le n\le 10^6$, $0\le k\le 10^3$).

\OutputFile
Выведите единственное целое число~--- ответ на задачу по модулю $10^9 + 7$.

\Examples

\begin{example}
\exmpfile{example.01}{example.01.a}%
\exmpfile{example.02}{example.02.a}%
\end{example}

\end{problem}

