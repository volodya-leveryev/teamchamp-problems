\begin{tutorial}{Радикал}

Обозначим для числа $N$ как его <<остаток>> $r=\dfrac{N}{\text{rad}(N)}$. Заметим, что при каждом изменении исходного числа:
\begin{enumerate}
\item Увеличивается на 1. Например:
$$N=12,\quad \text{rad}(N)=6,\quad r=2\qquad\rightarrow\qquad N=18,\quad \text{rad}(N)=6,\quad r=2+1=3.$$
\item Потом делится на все свои простые делители, которых не было у $N$. ГНапример:
$$N=24,\quad \text{rad}(N)=6,\quad r=4\qquad\rightarrow\qquad N=30,\quad \text{rad}(N)=30,\quad r=(4 + 1) \div 5=1.$$
\end{enumerate}

В языках программирования с длинной арифметикой достаточно просимулировать алгоритм, вычисляя радикал за разумное время $O(\min(\sqrt{N},\log{N}))$. Для длинных чисел радикал можно вычислить за логарифм, так как самым большим простым делителем в разложении может добавиться только $k$-ое простое число, как было показано выше.

Для языков без длинной арифметики можно хранить множество простых делителей, остаток $r$ и проверять с каждым новым изменением, не появляются ли у $r$ новые делители.

\end{tutorial}
