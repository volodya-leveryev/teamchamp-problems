\begin{problem}{Каверна}{стандартный ввод}{стандартный вывод}{1 секунда}{256 мегабайт}

<<Хитрые зайцы>> попали в опасную каверну и должны как можно быстрее выбраться из неё. Каверна представляет собой прямоугольное клеточное поле размером $N\times M$, где каждая клетка может:
\begin{itemize}
\item быть пустой;
\item содержать препятствие, через которое нельзя пройти;
\item содержать опасных эфириалов, с которыми необходимо сразиться при входе в клетку.
\end{itemize}

Помогите <<хитрым зайцам>> выбраться из каверны с минимальным числом сражений, если они могут передвигаться только по соседним по стороне клеткам. Они начинают в клетке $(1,1)$ и должны добраться до клетки $(N,M)$.

\InputFile
Первая строка содержит два целых числа $N$ и $M$ ($1\le N\cdot M\le10^5$)~--- размеры каверны.

Далее следуют $N$ строк длины $M$, описывающие каверну:
\begin{itemize}
\item символ <<\texttt{.}>> обозначает пустую строку;
\item символ <<\texttt{\#}>> обозначает препятствие;
\item символ <<\texttt{X}>> обозначает клетку с эфириалом.
\end{itemize}
Гарантируется, что клетки $(1,1)$ и $(N,M)$ всегда пустые.

\OutputFile
Выведите единственное целое число~--- минимальное количество сражений с эфириалами, чтобы дойти от клетки $(1,1)$ до клетки $(N,M)$. Если пути не существует, выведите $-1$.









\Examples

\begin{example}
\exmpfile{example.01}{example.01.a}%
\exmpfile{example.02}{example.02.a}%
\exmpfile{example.03}{example.03.a}%
\end{example}

\end{problem}

