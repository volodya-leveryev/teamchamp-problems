\begin{tutorial}{Чиселки}

Самым маленьким числом среди всех чисел с заданной суммой цифр является то, которое имеет в первую очередь минимальную длину. Это достигается размещением наибольших возможных цифр (девяток) в младших разрядах. Следовательно, чиселка с суммой цифр $S$ --- это наименьшее число, сумма цифр которого равна $S$, при этом оно должно состоять из максимального количества цифр <<9>> в младших разрядах, а самый старший разряд --- это остаток от деления $S$ на 9.

\end{tutorial}
