\begin{problem}{Чиселки}{стандартный ввод}{стандартный вывод}{1 секунда}{256 мегабайт}

Назовём натуральное число \textit{чиселкой}, если оно является наименьшим числом среди всех чисел с той же суммой цифр. Ваша задача~--- найти $N$-ую чиселку среди всех натуральных чисел.

\InputFile
Единственная строка содержит единственное натуральное число $N$ ($1\le N\le 1000$).

\OutputFile
Выведите единственное натуральное число~--- ответ на задачу.

\Examples

\begin{example}
\exmpfile{example.01}{example.01.a}%
\exmpfile{example.02}{example.02.a}%
\end{example}

\end{problem}

