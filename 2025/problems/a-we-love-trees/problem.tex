\begin{problem}{Мы обожаем деревья}{стандартный ввод}{стандартный вывод}{2 секунды}{256 мегабайт}

Мы обожаем деревья! Но больше всего их обожает Алёша. Недавно он придумал игру на дереве, которой он очень сильно хочет поделиться!

Дано дерево (связный граф без циклов) с $n$ вершинами и корнем в вершине $1$. У игрока имеется фишка, с помощью которой он делает ходы. Игру можно описать следующим образом:

\begin{enumerate}
\item Сначала игрок выбирает любую вершину в качестве начальной и ставит на неё фишку.
\item За один ход игрок может переместить фишку либо к родителю, либо к родителю родителя. При этом имеются две особенные вершины $u$ и $v$: в вершине $u$ имеется портал, который дополнительно может телепортировать фишку в вершину $v$. После использования портала он ломается, то есть портал можно использовать в игре максимум $1$ раз.
\item Ход фишки в $i$-ю вершину изменяет очки игрока на величину $a_i$. Если вершина была посещена повторно, то $a_i$ снова прибавляется к очкам. Начальная вершина тоже сразу же, как на неё поставлена фишка, приносит очки.
\item Игра заканчивается, когда игрок доходит до корня.
\end{enumerate}

Алёша решил сделать $m$ запросов, которые характеризуются вершинами $u$ и $v$. Ответом на каждый запрос является максимальное количество очков, которое игрок может набрать, если портал стоит в вершине $u$ и ведёт в вершину $v$. Требуется вывести ответ на каждый запрос.

\InputFile
Первая строка содержит два целых числа $n$ и $m$ ($3 \leq n \leq 10^5; 1 \leq m \leq 10^5$)~--- количество вершин и запросов соответственно.

Вторая строка содержит $n-1$ целых чисел $p_2,p_3,...,p_n$ ($1 \leq p_i \leq n$)~--- каждое $p_i$ является родителем вершины с номером $i$.

Третья строка содержит $n$ целых чисел $a_1,a_2,...,a_n$ ($-10^4 \leq a_i \leq 10^4$)~--- значения очков $i$-й вершины.

Затем следуют $m$ строк, каждая из которых содержит два целых числа $u$ и $v$ ($1 \leq u,v \leq n; u \neq v; u \neq 1$)~--- местоположение портала и куда он ведёт соответственно.

\OutputFile
Построчно для каждого запроса выведите ответ (в $m$ строках).

\Example

\begin{example}
\exmpfile{example.01}{example.01.a}%
\end{example}

\Note
Рассмотрим первый запрос. Траектория движения следующая: $5 \rightarrow 4 \rightarrow 2 \rightarrow$ (телепорт) $\rightarrow 4 \rightarrow 1$.

\end{problem}

