\begin{tutorial}{Мы обожаем деревья}

Давайте сначала решим задачу для дерева без портала. Заметим, что ответом для вершины $i$ является максимум из $0$, ответов потомков и ответов потомков потомков, к которому прибавлен $a_i$. Данные ответы будем хранить в массиве \texttt{dp\_down[n]}.

Далее, заметим, что при введении портала, ответом для вершины $1$ является либо \texttt{dp\_down[1]}, либо счёт, который получается, если мы оптимально дойдём до портала, воспользуемся им и оптимально дойдём до вершины $1$. Пусть портал находится в вершине $u$ и ведёт в вершину $v$. Перед использованием портала максимальный счёт равен \texttt{dp\_down[u]}. Пусть максимальным счётом от вершины $v$ до вершины $1$ является \texttt{dp\_up[v]}. Теперь надо понять, как заполнить этот массив.

Давайте перефразируем задачу для заполнения \texttt{dp\_up[n]}. Мы начинаем в вершине $1$ и можем переместиться либо в потомка, либо в потомка потомка. Базой \texttt{dp\_up[n]} является \texttt{dp\_up[1]} = $a_1$. Далее, ответом для \texttt{dp\_up[i]} является максимум из ответов предков и ответов предков предков, к которому прибавлен $a_i$. Заметьте, что \texttt{dp\_up[i]} может быть меньше $a_i$, так как мы уже не можем пропускать вершину от вершины $1$ до вершины $i$.

Наконец, ответом для запроса $u,v$ является \texttt{max(dp\_down[1], dp\_down[u] + dp\_up[v])}.

\end{tutorial}
