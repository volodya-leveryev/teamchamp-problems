\begin{problem}{Тимур любит куличи}{стандартный ввод}{стандартный вывод}{1 секунда}{256 мегабайт}

Совсем скоро будет Пасха, и Тимур очень ждет этот праздник. Особенно он любит пасхальные куличи.

Куличи бывают разных размеров, для простоты будем считать, что каждый кулич можно представить в виде прямоугольника на плоскости с целочисленными координатами вершин и сторонами, параллельными осям координат.

В стране, где он живет, принято разрезать куличи по диагонали на две равные половинки. Тогда половинки представляют собой прямоугольные треугольники с целочисленными координатами вершин и катетами, параллельными осям координат.

Тимуру подарили две такие половинки от куличей. Он задался вопросом: можно ли эти половинки совместить в целый кулич с помощью параллельного переноса? Вращать половинки при этом \textbf{нельзя}.

Он дал вам координаты вершин двух половинок и просит помочь найти ответ на вопрос.

\InputFile
В первой строке заданы 6 целых чисел $x_1$, $y_1$, $x_2$, $y_2$, $x_3$, $y_3$~--- соответственно, координаты первой, второй и третьей вершин треугольника, представляющего первую половинку кулича Тимура. \textbf{Гарантируется}, что вершина с~координатами $x_1$, $y_1$ лежит напротив гипотенузы.

Во второй строке заданы 6 целых чисел $x_4$, $y_4$, $x_5$, $y_5$, $x_6$, $y_6$~--- соответственно, координаты первой, второй и третьей вершин треугольника, представляющего вторую половинку кулича Тимура. \textbf{Гарантируется}, что вершина с координатами $x_4$, $y_4$ лежит напротив гипотенузы.

Все координаты не превышают $10^9$ по модулю.

\OutputFile
Выведите <<YES>>, если две данные половинки могут образовать один кулич, в противном случае выведите <<NO>>.

\Examples

\begin{example}
\exmpfile{example.01}{example.01.a}%
\exmpfile{example.02}{example.02.a}%
\end{example}

\end{problem}

