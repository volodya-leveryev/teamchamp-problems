\begin{problem}{RAID 5}{стандартный ввод}{стандартный вывод}{1 секунда}{256 мегабайт}

Имеется сообщение из $N$ символов. Каждый символ в нем имеет свой числовой код. Допустимыми символами являются:
\begin{itemize}\setlength\itemsep{0pt}
\item цифры (код совпадает со значением цифры);
\item строчная буква латинского алфавита (коды имеют значения в диапазоне 10--35);
\item заглавная буква латинского алфавита (коды имеют значения в диапазоне 36--61);
\item подчеркивание (код символа 62);
\item точка (код символа 63).
\end{itemize}

Сообщение разбивается на $K$ блоков по следующему принципу: первый символ записывается в первый блок, второй символ записывается во второй блок и~т.~д. $(K+1)$-й символ снова записывается в первый блок, $(K+2)$-й символ --- во второй блок и~т.~д.

Все блоки должны иметь одинаковое количество символов, обозначаемое как $L$. Если в блоке не хватает символа, тогда в него следует дописать символ <<точка>> (с кодом 63).

Например, сообщение <<\texttt{Hello\_World}>> разделяется на 3 блока следующим образом: <<\texttt{HlWl}>>, <<\texttt{eood}>> и <<\texttt{l\_r.}>>.

Для повышения надежности к блокам сообщения добавляется ещё один блок такой же длины. Каждый символ в добавочном блоке подбирается таким образом, чтобы сумма кодов всех символов на $i$-й позиции в разных блоках была кратна 64.

То есть, дополнительный блок для предыдущего примера равен <<\texttt{Oljv}>>. Если сложить коды первых символов в каждом блоке, получим: $43(H) + 14(e) + 21(l) + 50(O) = 128$. Если сложить коды вторых символов в каждом блоке, получим: $21(l) + 24(o) + 62(\_) + 21(l) = 128$.

В процессе передачи блоков сообщения один из них был утерян. Напишите программу, которая сможет восстановить его.


\InputFile
В первой строке входных данных заданы $K$ и $L$~--- целые положительные числа, разделенные пробелом ($0 \leq K, L \leq 10^2$).

В следующих $K$ строках заданы известные блоки сообщения.

\OutputFile
Выведите символы недостающего блока сообщения.

\Example

\begin{example}
\exmpfile{example.01}{example.01.a}%
\end{example}

\end{problem}

