\begin{problem}{Номера квартир}{стандартный ввод}{стандартный вывод}{1 секунда}{256 мегабайт}

Лёлек и Болек работают на стройке. На завершающей стадии строительства нового многоквартирного дома им поручили нанести краской на дверях квартир номера. В первой квартире дома расположится детский сад, на дверях которого будет красивая табличка, так что туда решили ничего не наносить. Бумажные трафареты для краски одноразовые, и~каждый содержит ровно одну цифру, так что нужно заранее точно посчитать их количество. Зная число квартир $N$ в доме, определите, сколько всего трафаретов с каждой из цифр потребуется? Лёлек и Болек -- шустрые ребята, и могут успеть много. Поэтому выведите не сами величины, а их остатки от деления на $10^9$.

\InputFile
В первой строке входного файла содержится одно натуральное число $N$ ($1 \leq N \leq 10^{1000}$).

\OutputFile
В выходной файл вывести через пробел десять чисел $M_0$, $M_1$, \dots, $M_9$, где $M_k$~--- остаток от деления числа появлений цифры $k$ в нумерации квартир на
$10^9$, $k = 0, 1, \ldots , 9$.

\Example

\begin{example}
\exmpfile{example.01}{example.01.a}%
\end{example}

\end{problem}

