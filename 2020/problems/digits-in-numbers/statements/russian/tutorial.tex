\begin{tutorial}{Номера квартир}

Подсчитаем, сколько раз в номерах квартир определенная цифра $k$ появляется в какой-нибудь заданной позиции $p$. Эта величина зависит от цифр самого числа
$N = d_P \ldots d_2 d_1$. Начнем рассмотрение с примера.

Пусть $N =\,$\texttt{742\underline{6}310}, и нас интересует (подчеркнута) четвертая справа позиция. Количество вхождений во все числа от 1 до $N$ в этой позиции $p=4$ будет различным для разных цифр $k$. Будем отдельно интересоваться, что может быть записано слева от данной позиции, обозначая эту часть ???, и справа, обозначая правую часть звездочками. Сведем все случаи в таблицу:\\[2mm]
\centerline{
\begin{tabular}{r||r|r|r|r|}
Цифра $k$ &
\multicolumn{1}{c|}{$0$} &
\multicolumn{1}{c|}{$1\leqslant k<6$} &
\multicolumn{1}{c|}{$k>6$} &
\multicolumn{1}{c|}{$k=6$} \\
\hline
\hline
Входит  & от  \texttt{1\underline{0}***} &  от \texttt{\underline{3}***} &   от \texttt{\underline{8}***}& от \texttt{\underline{6}***}\\
в числа &до \texttt{742\underline{0}***} &до \texttt{742\underline{3}***}&до \texttt{741\underline{8}***}&до \texttt{741\underline{6}***} \\
\cline{2-5}
& & & & и от \texttt{742\underline{6}000}\\
& & & & до \texttt{742\underline{6}310}\\
\end{tabular}
}

Выделим числа $L_p$ и $R_p$, образованные цифрами числа $N$ до и после выбранной позиции $p=4$:
\\[1mm]
\centerline{
\begin{tabular}{c|c@{}|@{}c@{}|@{}c|}
$N = \,$ &
\texttt{742}&\underline{\texttt{6}}&\texttt{310}\\
\cline{2-4}
\multicolumn{1}{c}{} &
\multicolumn{1}{l@{}}{$L_4$} &
\multicolumn{1}{@{}c@{}}{$d_4$} &
\multicolumn{1}{@{}r}{$R_4$}
\end{tabular}
}
Вместо каждой из звездочек справа может стоять любая цифра, то есть вариантов заполнения звездочек $10^3 = 10^{p-1}$. Варианты заполнения вопросительных знаков слева зависят от числа $L_4=742$. Кроме того, если слева стоит само $L_p$, а в выбранной позиции цифра $d_p=$\,\underline{\texttt{6}}, то продолжить такое число справа можно только записями (с ведущими нулями) чисел от 0 до $R_4=310$.

Обозначим через $m_p(k)$ общее число вхождений цифры $k$ в $p$-ю позицию в числа от 1 до $N$. Обобщая проведенное рассмотрение, получаем, что для $1 < p < P$:
$$
m_p(k)
 =\left\{
 \begin{array}{ll}
L_p \times 10^{p-1}, & k=0\\
(L_p + 1) \times 10^{p-1}, & k < d_p,\\
L_p \times 10^{p-1} + (R_p + 1), & k = d_p,\\
L_p \times 10^{p-1}, & k > d_p
\end{array}
\right.
$$
Доработав эту формулу для крайних разрядов, и просуммировав для каждой цифры по всем позициям, а также не забыв вычесть единицу для первой квартиры, получим решение. Все вычисления следует проводить по модулю $10^9$, поэтому при $p>1$ ненулевыми будут только вклады от слагаемых вида $R_p+1$.


\end{tutorial}
