\begin{problem}{Поисковая система}{стандартный ввод}{стандартный вывод}{2 секунды}{256 мегабайт}

Молодой якутский натуралист Хатан продолжает изучение животного мира. За все время исследований он уже накопил множество разных записей, в которых содержатся интересные факты и наблюдения о животных. Он хочет организовать поиск по накопленным данным, чтобы быстрее получать ответы на интересующие его вопросы. Хатан уже придумал метод оценки записей для поиска, но его навыки программирования не позволяют самому решить эту задачу. 

Результатом поиска по введенным $N$ записям и запросу является запись с максимальной <<релевантностью>>~--- величиной, определяющей насколько та или иная запись соответствует запросу.

Каждая запись и запрос представляют собой последовательность слов, состоящих из английских букв и цифр и разделённых пробелами.

<<Релевантностью>> записи называется целое число, которое вычисляется сравнением каждой пары слов из записи и слов из запроса с учетом следующих условий:
\begin{itemize}
 \item Если слово из запроса содержится в слове из записи, то к <<релевантности>> прибавляется $A*\text{<длина слова из запроса>}$;
 \item Если слово из запроса является префиксом слова из записи, то к <<релевантности>> прибавляется $B*\text{<длина слова из запроса>}$;
 \item Если слово из запроса является суффиксом слова из записи, то к <<релевантности>> прибавляется $C*\text{<длина слова из запроса>}$.
\end{itemize}

Эти условия не исключают друг друга, то есть, например, если слово из запроса полностью совпадает со словом из записи, то оно учитывается и как префикс, и как суффикс, и как содержащееся.

Записи Хатан ведет в хронологическом порядке и считает, что актуальность данных превыше всего, поэтому если <<релевантность>> некоторых записей совпадает, то необходимо вывести запись, которая во входных данных появляется последней.

При сравнении строк регистр букв не учитывается.

Гарантируется, что <<релевантность>> помещается в 32-битный тип данных \texttt{int}.

В запросе и записях могут встречаться одинаковые слова~--- их нужно учитывать по отдельности.

\InputFile
В первой строке записан запрос в поисковую систему. Во второй строке~--- три целых числа $A$, $B$, $C$, разделенные пробелами ($0 \leq A,B,C \leq 100$). В третьей строке задано число $N$~--- количество записей, по которым необходимо производить поиск ($1 \leq N \leq 300$). Далее идут $N$ строк, где каждая строка является одной записью. Длина каждой из строк не превышает 300 символов.

\OutputFile
Необходимо вывести наиболее <<релевантную>> запись в таком же виде, в каком она введена во входных данных.

\Examples

\begin{example}
\exmpfile{example.01}{example.01.a}%
\exmpfile{example.02}{example.02.a}%
\end{example}

\Note
Рассмотрим первый тест: 
\begin{itemize}
\item Запрос~--- \texttt{\emph{How} \emph{much} \emph{do} \emph{puppies} \emph{weigh} \emph{in} \emph{kg}}
\item Первая запись~--- \texttt{\emph{puppies}} совпадает со словом из запроса, то есть содержит его, имеет суффикс и префикс равный ему. Итоговая <<релевантность>>~--- $7*(1+1+1) = 21$
\item Вторая запись~--- \texttt{\emph{puppies}} и \texttt{\emph{in}}~--- два слова полностью совпадают со словами из запроса. Итоговая <<релевантность>>~--- $7*(1+1+1) + 2*(1+1+1) = 27$
\item Третья запись~--- \texttt{\emph{puppies}} полностью совпадает, а \texttt{amaz\emph{in}g} содержит слово \texttt{in}. Итоговая <<релевантность>>~--- $7*(1+1+1) + 2*(1+0+0) = 23$
\item Четвертая запись~--- \texttt{\emph{puppies}}, \texttt{\emph{weigh}} и \texttt{\emph{kg}} полностью совпадают со словами из запроса. Итоговая <<релевантность>>~--- $7*(1+1+1) + 5*(1+1+1) + 2*(1+1+1) = 42$
\end{itemize}

\end{problem}

