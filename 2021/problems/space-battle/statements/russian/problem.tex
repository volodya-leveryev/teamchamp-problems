\begin{problem}{Космический бой}{стандартный ввод}{стандартный вывод}{1 секунда}{256 мегабайт}

Вася и Коля играют в~трехмерный космический бой~--- аналог двухмерного морского боя. Игра ведётся на трехмерном клеточном поле, состоящем из объёмных кубических клеток. Каждая клетка обозначается тремя целыми числами, каждое из которых обозначает номер клетки по соответствующей координатной оси трехмерного космического пространства.

Множество клеток, имеющих общие грани, называется кораблём. Две клетки принадлежат одному и тому же кораблю, если между ними существует непрерывная последовательность из клеток с~общими гранями.

Две клетки НЕ принадлежат одному и тому же кораблю, если они имеют общее ребро или вершину, но НЕ имеют непрерывной последовательности клеток с~общими гранями.

Однажды случилась очень странная ситуация: Вася начинал игру и, какую бы клетку он не назвал, она непременно попадала в корабль противника. Через определенное число ходов Вася выиграл, а Коле не удалось сделать и хода. Определите, сколько кораблей используется в~данной игре.

\InputFile
В~первой строке входных данных указывается количество ходов $N$, сделанных Васей ($1 \leq N \leq 3 \cdot 10^4$).

В~следующих $N$ строках даются по три целых положительных числа $x$, $y$, $z$ ($1 < x, y, z \leq 10^4$), разделённых пробелами. Каждая такая тройка обозначает клетку, принадлежащую кораблю Коли (и пораженную при очередном выстреле Васи).

Понятно, что Вася не всегда называл клетки, располагающиеся подряд и не стрелял в уже пораженную клетку.

\OutputFile
Программа должна вывести искомое количество кораблей, которое было на игровом поле Коли.

\Examples

\begin{example}
\exmpfile{example.01}{example.01.a}%
\exmpfile{example.02}{example.02.a}%
\end{example}

\end{problem}

