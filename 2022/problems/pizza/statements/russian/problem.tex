\begin{problem}{Калькулятор пиццы}{стандартный ввод}{стандартный вывод}{1 секунда}{256 мегабайт}

Молодой программист Айтал получил заказ от сайта <<Каталог калькуляторов>> на разработку нового калькулятора пиццы. Пусть пиццерия предлагает пиццы трех разных размеров, известны их диаметры и цены, а покупатель желает приобрести несколько пицц общей площадью не менее $N$ $\mbox{см}^2$ и при этом сэкономить.

Калькулятор должен определить минимальную сумму денег, которую придется потратить покупателю.

\InputFile
В~первой строке задано целое число $N$~--- желаемая покупателем суммарная площадь пицц ($1 \leq N \leq 10^5$).

В~трех следующих строках через пробел заданы по два целых числа: соответственно, диаметр $d_i$ и цена $p_i$ каждой $i$-й пиццы ($1 \leq d_i, p_i \leq 10^5$, $i = 1, 2, 3$).

\OutputFile
Программа должна вывести одно целое число~--- ответ на задачу.

\Example

\begin{example}
\exmpfile{example.01}{example.01.a}%
\end{example}

\Note
В~примере нужно купить одну среднюю и одну маленькую пиццу. Их суммарная площадь будет равна 
$$\pi(25/2)^2 + \pi(30/2)^2 \approx 490.87 + 706.86 = 1197.73 > 1000.$$ Их общая стоимость: $545 + 829 = 1374$.

\end{problem}

