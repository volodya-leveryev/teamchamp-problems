\begin{problem}{Своя игра}{стандартный ввод}{стандартный вывод}{1 секунда}{256 мегабайт}

После очередной победы в тяжёлой игре, Ки Хун решил придумать свою игру. Участники игры должны пройти некоторый лабиринт от входа до выхода, собрав карточки у всех солдат, находящихся в лабиринте. Солдат отдает игроку свою карточку только в том случае, если игрок подошел к~нему по \textit{верному} маршруту. Маршрут от входа до выхода будет верным, если при условии обхода всех солдат, он будет наикратчайшим из всех возможных маршрутов. 

Ки Хун дал вам протестировать эту игру и просит написать программу, которая для произвольного лабиринта выведет количество солдат в нем и длину верного маршрута до каждого очередного солдата.

\InputFile
В первой строке записаны натуральные числа $N$ и $M$~--- размеры лабиринта ($3 \leq N, M \leq 10$).

Во второй строке записаны $i_s$ и $j_s$~--- координаты входа в лабиринт, в третьей строке записаны числа $i_e$ и $j_e$~--- координаты выхода из лабиринта ($0 \leq i_s, i_e < N$, $0 \leq j_s, j_e < M$). 

В последующих $N$ строках через пробел записаны по $M$ чисел, описывающих лабиринт: 0~--- свободная клетка, 1~--- стена, 2~--- солдат. 

Солдат в лабиринте может быть не более 10.

\OutputFile
В первой строке вывести искомое количество солдат $K$. Затем вывести $K$ чисел~--- количество клеток лабиринта, пройденного до каждого очередного солдата по верному маршруту.

\Example

\begin{example}
\exmpfile{example.01}{example.01.a}%
\end{example}

\Note
Гарантируется, что каждый солдат достижим для участников игры, и в лабиринте обязательно есть хотя бы один солдат. Также гарантируется, что идя по кратчайшему маршруту нельзя встретить одного и того же солдата более одного раза, а также гарантируется то, что верный маршрут~--- единственный.

В примере солдаты имеют следующие координаты: (2, 0), (3, 2) и (1, 4).

\end{problem}

