\begin{tutorial}{Максим и плоская планета}

Для решения данной задачи нужно воспользоваться алгоритмом Грэхема построения выпуклой оболочки в двумерном пространстве. Данный алгоритм позволяет построить выпуклую оболочку за $O(n \mbox{ log } n)$.

\begin{enumerate}
\item Находим точку $p_0$ нашего множества с самой маленькой у-координатой (если таких несколько, берем самую правую из них), добавляем в ответ.
\item Сортируем все остальные точки по полярному углу относительно $p_0$.
\item Добавляем в ответ $p_1$ --- самую первую из отсортированных точек.
\item Берем следующую по счету точку $t$. Пока $t$ и две последних точки в текущей оболочке $p_i$ и $p_{i-1}$ образуют неправый поворот (векторы $p_i t$ и $p_{i-1} p_i$), удаляем из оболочки $p_i$.
\item Добавляем в оболочку $t$.
\item Повторяем пункты 4-5, пока не закончатся точки.
\end {enumerate}


Первым делом находим выпуклую оболочку для всех точек. Получаем список точек, которые входят в выпуклую оболочку. Очевидно, что интересующая нас точка находится именно в этом списке. Рассмотрим одну из точек списка. Назовем ее $curr$, предыдущую и следующую за ним точки $prev$ и $next$ соответственно. Теперь нужно повторить пункты 4-5 алгоритма Грэхема с того момента, когда была добавлена точка $prev$, только игнорируя точку $curr$. Останавливаемся, когда добавим точку $next$. Таким образом, мы получим выпуклую оболочку без точки $curr$. И для каждой такой точки можем высчитать на сколько уменьшается площадь. Выбираем точку, которая уменьшает площадь на наибольшее значение. 

\end{tutorial}
