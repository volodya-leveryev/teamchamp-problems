\begin{problem}{Максим и плоская планета}{стандартный ввод}{стандартный вывод}{1 секунда}{256 мегабайт}

Максим очень сильно увлекается плоскими планетами. Однажды, изучая одну из таких планет через сверхточный телескоп, он заметил интересное государство. В этом государстве находятся $N$~городов. Как он заметил, территорией государства является наименьший выпуклый многоугольник, содержащий все города этого государства.

Максим видит, что на этой планете часто идут территориальные войны. Он хочет предупредить правительство данного государства о том, какой из городов является самым привлекательным для вражеских государств, а именно, захват какого города приведет к~тому, что площадь оставшейся территории государства станет минимальной.

Максим не умеет программировать в отличие от вас. Он просит вас помочь определить такой город. С~помощью подробной карты, предоставленной Максимом, укажите номер искомого города.


\InputFile
В первой строке находится целое число $N$ $(3 \leq N \leq 10^5)$ --- количество городов. В следующих $N$ строках даны целые координаты городов $x_i$ $y_i$ $(0 \leq x_i, y_i \leq 10^6)$. Никакие два города не находятся в одной и той же точке. Никакие три города не лежат на одной прямой.

\OutputFile
Выведите номер города. Города нумеруются в том порядке, в котором они заданы во входных данных, начиная с 1. Если несколько городов удовлетворяют решению задачи, выведите наименьший возможный номер.

\Example

\begin{example}
\exmpfile{example.01}{example.01.a}%
\end{example}

\end{problem}

