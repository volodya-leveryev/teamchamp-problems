\begin{problem}{Одномерный гектар}{стандартный ввод}{стандартный вывод}{3 секунды}{256 мегабайт}

В некотором одномерном государстве решили каждому жителю раздать по гектару земли. Каждая семья государства выбрала самый лучший участок и подала заявку на выбранный участок. Оказалось, что некоторые из выбранных участков могли пересекаться или совпадать. Помогите главному землеустроителю выбрать из поданных заявок максимальное количество непересекающихся участков (не имеющих общих точек).

\InputFile
В первой строке входного файла задано целое число $N$ ($1\leqslant N \leqslant 10^6$). В~каждой из последующих $N$ строк~--- пара целых чисел $a_i$, $b_i$, $i=1,2,\ldots,N$ ($-1000\leqslant a_i < b_i \leqslant 1000$), задающих координаты левого и правого концов участка.

\OutputFile
Выходной файл должен в первой строке содержать найденное количество участков, а затем список выбранных участков. Описание каждого участка должно располагаться на отдельной строке в виде пары чисел, задающих левый и правый концы участка. Если есть несколько решений, можно напечатать любое из них.

\Example

\begin{example}
\exmp{3
-2 4
4 5
5 8
}{2
-2 4
5 8
}%
\end{example}

\end{problem}

