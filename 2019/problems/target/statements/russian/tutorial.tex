\begin{tutorial}{Мишень}

Для каждого выстрела по его координатам вычислим расстояние до центра мишени. Координаты центра мишени по условию равны ($N$, $K$). Выделим целую часть половины вычисленного расстояния и обозначим через $d$. Если половина расстояния целое число, то пуля попала на границу зон, $d$~уменьшим на 1. Тогда $d$ будет равно номеру концентрических окружностей мишени, т.е. $10-d$ равно очкам за выстрел. Просуммируем полученные очки.


\end{tutorial}
