\begin{tutorial}{Бисер}

Обозначим через $f[i][j]$ количество всевозможных наборов для $i$ бисеринок и $j$ пакетиков. Очевидно, что $f[i][i] = f[i][1] = 1$ для всех $i$. 

Получим рекуррентное соотношение для определения $f[i][j]$. Пусть мы имеем решение $f[i-1][j]$ для $i-1$ бисеринок и $j$ пакетиков. Ответ для $i$ бисеринок и $j$ пакетиков можно получить из набора $f[i-1][j]$ следующими способами: 
\begin{itemize}
\item положив в каждый пакетик по одной бисеринке, получим  $j \times f[i-1][j]$ вариантов ответа,
\item положив в один последний пакетик только одну бисеринку, что даст $f[i-1][j-1]$ вариантов. 
\end{itemize}
Имеем $f[i-1][j]= f[i-1][j-1]+ j \times f[i-1][j]$.


\end{tutorial}
