\begin{problem}{Бисер}{стандартный ввод}{стандартный вывод}{1 секунда}{256 мегабайт}

У Лены мама очень любит делать украшения из японского бисера. Лена решила на 8 марта подарить ей $K$ пакетиков бисера. Для этого она купила в интернет-магазине $N$ разноцветных бисеринок. Сколько вариантов подарка может получиться у Лены? 



\InputFile
Входной файл содержит целые числа $N$ и $K$ ($1 \le K \le N \le 20$). 

\OutputFile
Требуется вывести в выходной файл единственное целое число --- искомое число вариантов.

\Examples

\begin{example}
\exmpfile{example.01}{example.01.a}%
\exmpfile{example.02}{example.02.a}%
\end{example}

\Explanations
В первом примере только один пакет, поэтому имеем один набор \{[1,2,3,4]\}, где бисеринки разного цвета обозначены цифрами от 1 до 4.
 
Во втором примере имеем два пакета, тогда возможны следующие наборы: \{[1],[2,3,4]\}, \{[2],[1,3,4]\}, \{[3],[1,2,4]\}, \{[4],[1,2,3]\}, \{[1,2],[3,4]\}, \{[1,3],[2,4]\}, \{[1,4],[2,3]\}.


\end{problem}

