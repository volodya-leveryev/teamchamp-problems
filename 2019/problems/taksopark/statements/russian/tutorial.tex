\begin{tutorial}{Таксопарк}

Рассмотрим двудольный граф, в котором одной доле сопоставим водителей, а другой -- автомобили. Ребро между вершиной $i$ из первой доли и вершиной $j$ из второй будет обозначать, что водитель с номером $i$ имеет допуск к автомобилю с номером $j$. Требуется выбрать для каждой вершины из первой доли вершину из второй, причём разным вершинам из первой доли должны соответствовать разные вершины из второй доли.\\
Возьмём любую компоненту связности в этом двудольном графе. Степень каждой вершины в этой компоненте чётна, а значит, в этой компоненте существует эйлеров цикл (цикл, в котором каждое ребро проходится ровно один раз). Удалим из нашего графа ребра этого цикла. После такой операции степень каждой вершины станет равна $2$. Повторим этот процесс второй раз и удалим из нашего графа каждое второе ребро в этом цикле, тогда степень каждой вершины станет равна $1$ -- каждый водитель имеет допуск ровно к одному автомобилю. Требуемый допуск определятся однозначно.\\
Проведём эту операцию с каждой компонентой связности.


\end{tutorial}
